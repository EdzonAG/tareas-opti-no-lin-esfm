%-------------------------------------------------%
\documentclass[12pt,oneside]{book}
%-------------------------------------------------%
\usepackage[utf8x]{inputenc}
\usepackage[spanish]{babel}
\usepackage[margin=1in]{geometry}
\usepackage{graphicx}
\usepackage{amssymb,amsmath,amsthm,amsfonts}
\usepackage{enumerate}
\usepackage{lipsum}
\usepackage{color}
\usepackage{float}
\usepackage{parskip}
\usepackage{titlesec}
\usepackage{apacite}
\usepackage{listings}
\usepackage{color}
\definecolor{mygreen}{RGB}{28,172,0}
\definecolor{mylilas}{RGB}{170,55,241}
%-------------------------------------------------%
\titleformat{\chapter}{\normalfont\huge}{\thechapter.}{20pt}{\huge \bf}
\renewcommand{\qedsymbol}{\rule{0.7em}{0.7em}}
\providecommand{\abs}[1]{\lvert#1\rvert}
\graphicspath{ {Img/} }
\decimalpoint
%-------------------------------------------------%
\lstset{language=Matlab,%
    %basicstyle=\color{red},
    breaklines=true,%
    morekeywords={matlab2tikz},
    keywordstyle=\color{blue},%
    morekeywords=[2]{1}, keywordstyle=[2]{\color{black}},
    identifierstyle=\color{black},%
    stringstyle=\color{mylilas},
    commentstyle=\color{mygreen},%
    showstringspaces=false,%without this there will be a symbol in the places where there is a space
    numbers=left,%
    numberstyle={\tiny \color{black}},% size of the numbers
    numbersep=9pt, % this defines how far the numbers are from the text
    emph=[1]{for,end,break},emphstyle=[1]\color{red}, %some words to emphasise
    %emph=[2]{word1,word2}, emphstyle=[2]{style},    
}
%-------------------------------------------------%
\title{Tarea 2 - Optimización no Lineal}
\author{Camara Medina Cynthia Lilian \\
Reyes Zamora Ollin \\
Alanis González Edzon Omar}
\date{\today}
%-------------------------------------------------%
\begin{document}
\maketitle

\tableofcontents

\chapter{Teoría}

\section[Teorema de Taylor]{Enuncie el Teorema de Taylor para funciones de $\mathbb{R} \rightarrow \mathbb{R}$. Escriba la demostración o un esbozo de ésta.}

Sea $f$ continua en $[a,b]$ y con derivadas hasta de orden $n$ continuas también en este intervalo cerrado; supóngase que $f^{(n+1)}(x)$ existe en $(a.b)$ entonces para $x$ \& $x_{0\hat{i}}(a,b)$ se tiene

\begin{equation*}
    f(x) = f(x_0) + f'(x_0)(x-x_o) + \frac{f^{(2)}(x_0)}{2!}(x-x_o)^2 + ... + \frac{f^{(n)}(x_0)}{n!}(x-x_0)^n + E_n
\end{equation*}

donde $E_n = \frac{f^{(n+1)}(c)}{(n+1)!}(x-x_0)^{n+1}$ \& $c$ es un punto $[x,x_0]$

\begin{proof}
    Primero demostremos que si $p(x) = a_0 + a_1 x + a_2 x^2 + ... + a_n x^n$, entonces el polinomio también puede escribirse como

    \begin{equation*}
        p(x) = b_0 + b_1(x-x_0)+b_2(x-x_0)^2+...+b_n(x-x_0)^n
    \end{equation*}

    Procedemos por inducción

    para n = 0

    \begin{equation*}
        p(x) = a_0
    \end{equation*}

    para n = 1

    \begin{gather*}
        p(x) = a_0  + a_1 x = a_0 + a_1(x -x_0 + x_0)  = a_0 +a_1 x_0 +a_1(x-x_0) \\
        \Rightarrow p(x) = b_0+b_1(x-x_0)
    \end{gather*}

    ahora suponemos cierto para $n = k$, es decir se cumple que si

    \begin{equation*}
        p(x)  =a_0 + a_1x + a_2x^2 + ... + a_k x^k
    \end{equation*}

    entonces implica que

    \begin{equation*}
        p(x) = b_0 + b_1(x-x_0) + b_2 (x-x_0)^2 + ... + b_k (x-x_0)^k
    \end{equation*}

    ahora probemos que se cumple para $n = k+ 1$

    \begin{align*}
            p(x) &= a_0 + a_1x + a_2x^2 + ... + a_k x^k + a_{k+1}x^{k+1} \\
            &= a_0 + x(a_1 + a_2x + ... + a_k x^{k-1} + a_{k+1}x^k) \\
            &= a_0 + (x-x_0+x_0)(c_0+c_1(x-x_0) + ... +c_2(x-x_0)^k) \\
            &= a_0 + c_0(x-x_0)+c_1(x-x_0)^2+...+c_k(x-x_0)^{k+1}+ c_0x_0 + c_1x_0(x-x_0) \\
            &+ ... + c_k x_0(x-x_0)^k \\
            &= b_0 + b_1(x-x_0)+...+b_{k+1}(x-x_0)^{k+1}
    \end{align*}

    por lo que

    \begin{equation*}
        f(x) = p(x) = b_0 + b_1(x-x_0) + b_2 (x-x_0)^2 + ... + b_m (x-x_0)^m
    \end{equation*}

    Ahora tenemos que demostrar que $b_n = \frac{f^{(n)}(x_0)}{n!}, \quad \forall n = 0,1,2,...,m$

    \begin{gather*}
        f(x) = \sum_{n=0}^m b_n (x-x_0)^{n} \rightarrow f(x_0)=b_0 \\
        f'(x) = \sum_{n=1}^m n b_n(x-x_0)^{n-1} \rightarrow f'(x_0) = b_1 \\
        f''(x) = \sum_{n=2}^m n(n-1) b_n(x-x_0)^{n-2} \rightarrow f''(x_0) =2 \cdot 1 b_2 \rightarrow b_2 = \frac{f''(x_0)}{2!} \\
        f'''(x) = \sum_{n=3}^m n(n-1)(n-2)b_n(x-x_0)^{n-3} \rightarrow f'''(x)=3 \cdot 2 \cdot 1b_3 \rightarrow b_3 = \frac{f'''(x_0)}{3!} \\
        ...\\
        f^{(m)}(x) = \sum_{n= m}^m n (n-1)...(n-m+1)b_n(x-x_0)^{n-m} \rightarrow f^{(m)}(x_0) = m!\cdot b_m \rightarrow b_m = \frac{f^{(m)}(x_0)}{m!} \\
        f^{(m+1)}(x_0)= 0
    \end{gather*}
\end{proof}

\subsection[Explicación]{Explique con sus propias palabras y un ejemplo la importancia del
teorema anterior.}

Dado un polinomio cualquiera podemos expresarlo en potencias de $(x-x_0)$ para cualquier $x_0$.

Ejemplo:

\begin{equation*}
    7x^3+x^2+8 = 16+23(x-1)^2+22(x-1)^2+7(x-1)^3, \quad x_0=1
\end{equation*}

\begin{gather*}
    P(x) = 7x^3+x^2+8P(1)=16 \\
    P'(x) = 21x^2 + 2xP'(1) = 23 \\
    P''(x) = 42x + 2 P''(1) = 44 \\
    P'''(x) = 42 P'''(1) = 42
\end{gather*}

\begin{equation*}
    f(x) = 16+23(x-1) + \frac{44}{2!}(x-1)^2 + \frac{42}{3!}(x-1)^3
\end{equation*}

\section[Definiciones Conjuntos]{Escriba las definiciones de los conceptos de \textit{conjunto abierto, conjunto
cerrado} y \textit{vecindad}, para $\mathbb{R}^n$.}

\subsection{Conjunto Abierto}
Se incluyen todos los puntos comprometidos entre los limites, sin considerar los limites.

\subsection{Conjunto Cerrado}
Se incluyen todos los puntos comprometidos entre los limites, considerando los limites.

\subsection{Vecindad}
Conjunto que contiene al punto y a un conjunto de puntos cercanos a él.

$V$ es vecindad de $p$, si un disco alrededor de $p\in V$  

\section[Óptimo Local y Global $\mathbb{R}$]{Escriba las definiciones de los conceptos de \textit{óptimo local} y \textit{óptimo global}
de una función $f: \mathbb{R} \rightarrow \mathbb{R}$. Dé un ejemplo.}

\subsection{Óptimo Local}
$x_0$ es un óptimo local (extremo absoluto) si $\forall x \in f: \mathbb{R} \rightarrow \mathbb{R}$ cumple que $f(x_0) \geq f(x)$(máximo) o $f(x_0) \leq f(x)$(mínimo).

\subsection{Óptimo Global}
$x_0$ es un óptimo global (extremo relativo) si existe un entorno de $x_0$, $U(x_0)$ tal que  $\forall x \in U(x_0) \cap f: \mathbb{R} \rightarrow \mathbb{R}$ cumple que $f(x_0) \geq f(x)$(máximo) o $f(x_0) \leq f(x)$(mínimo).

\section[Óptimo Local y Global $\mathbb{R}^n$]{Escriba las definiciones de los conceptos de \textit{óptimo local} y \textit{óptimo global}
de una función $f: \mathbb{R}^n \rightarrow \mathbb{R}$. Dé un ejemplo.}

\subsection{Óptimo Local}
Un punto $x^* \in F$ se denomina un óptimo local del problema de optimización si
existe un entorno de $x^*$ tal que para cualquier otro punto $x \in F$ del entorno $J(x^*) \leq J(x)$ pueden existir varios mínimos locales en una misma región.

\subsection{Óptimo Global}
Un punto $x^*$ se denomina un óptimo global del problema de optimización si para cualquier punto $x$ del conjunto factible  $J(x^*) \leq J(x)$. Solamente existe un óptimo global en una función, debido a que este es el punto óptimo de la función.

\section[Gradiente de una función]{Escriba la definición del \textit{gradiente} de una función $f: \mathbb{R}^n \rightarrow \mathbb{R}$. Dé un
ejemplo.}

Es el vector cuyas proyecciones sobre los ejes de coordenadas son las correspondientes derivadas parciales de dicha función. Se denomina por $\nabla f$.

Ejemplo:

El gradiente de la función $f(x,y,z)$ en el punto $(x_0,y_0,z_0)$ es:

\begin{equation*}
    \left(\frac{\partial f}{\partial x} (x_0,y_0,z_0), \frac{\partial f}{\partial y}(x_0,y_0,z_0), \frac{\partial f}{\partial z}(x_0,y_0,z_0)\right)
\end{equation*}

Cada derivada parcial en el punto $(x_0,y_0,z_0)$ se llama componente del gradiente en ese punto.

\section[Derivada Direccional]{Escriba la definición de la derivada direccional de una función $f: \mathbb{R}^n \rightarrow \mathbb{R}$.
Dé un ejemplo.}

La derivada de una función, en la dirección de un vector dado, representa la
tasa de cambio de la función en la dirección de dicho vector. Este concepto
generaliza las derivadas parciales, puesto que estas son derivadas direccionales
según la dirección de los respectivos ejes coordenados.

\textit{Ejemplo:}
\begin{gather*}
    f(x,y) = -x^2-y^2+18  \\
    u = \cos{\left(\frac{3\pi}{4}\right)} \hat{i} + \sin{\left( \frac{3 \pi}{4}\right)} \hat{j}
\end{gather*}

\chapter{Programación}

{\large Escriba un programa, en Matlab/Octave, que tome como parámetros una función $f: \mathbb{R}^n \rightarrow \mathbb{R}$ y un punto $x_0 \in \mathbb{R}^n$ y devuelva una aproximación al vector gradiente 

\begin{equation*}
    \nabla f(x_0) \in \mathbb{R}^n
\end{equation*}

evaluado en el punto $x_0$.}

\begin{itemize}
    \item Lleve a cabo la programación de manera que el valor de $n$ no sea requerido
    como parámetro.
    \item El gradiente debe aproximarse numéricamente (no simbólicamente), es
    decir debe hacerse usando diferencias finitas.
    \item Muestre con un ejemplo numérico particular con $n = 4$, usando el gradiente calculado algebraicamente, que su programa aproxima correctamente los valores.    
\end{itemize}

\subsection{Código Matlab}

\textit{Ejemplo para la función} $f(x_n) = x_1x_2^3-x_3^4-x_4x_2$ en el punto $x_0 = (2,3,4,5)$
\begin{lstlisting}[frame=single]
    Ingrese la funcion:
    Use (x(1),x(2),...,x(n)) como variables

    f(x_n) = x(1)*x(2)^2-x(3)^4-x(4)*x(2)
    Ingresa el vector x0
    [2 3 4 5]

    El vector gradiente de la funcion:
    f(x_n)=x(1)*x(2)^2-x(3)^4-x(4)*x(2)
    es:
        9.0000    7.0000 -256.1600   -3.0000
\end{lstlisting}
\newpage

\textit{Ejemplo para la función} $f(x_n) = x_1^3+x_2^5x_5^4-2x_1^7-8x_3x_4^2$ en el punto $x_0 = (7,6,5,4,3)$

\begin{lstlisting}[frame=single]
    Ingrese la funcion f
    Use (x(1),x(2),...,x(n)) como variables

    f(x_n) = x(1)^3 + x(2)^5*x(5)^4-2*x(1)^7-8*x(3)*x(4)^2
    Ingresa el vector x0
    [7 6 5 4 3]

    El vector gradiente de la funcion:
    f(x_n)=x(1)^3 + x(2)^5*x(5)^4-2*x(1)^7-8*x(3)*x(4)^2
    es:
    1.0e+06 *

        -1.6486    0.5252   -0.0001   -0.0003    0.8407
\end{lstlisting}

\end{document}